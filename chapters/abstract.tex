% -*- root: diplomarbeit.tex -*-
\begin{abstract}
\begin{singlespace}
In dieser Arbeit wird ein Tracking-Detection-Algorithmus hinsichtlich seiner Eignung für das Finden und Verfolgen von Flugroboter, wie Quadro- und Hexakopter, untersucht. Im Detail besteht die Aufgabe, dass ein solcher Roboter ausgestattet mit einer Kamera, visuellen Informationen für das Finden eines anderen Roboters verarbeitet. 

Der untersuchte Ansatz TLD - Tracking Learning Detection \cite{TLD} ist ein semi-automated single-target tracking-Algorithmus. Das single-target tracking schätzt in einer Sequenz von Bildern $I_{0},\dots,I_{m}$ den Status $x_{k}$ eine Objekts im Bild $I_{k}$ \cite{key-7}. Der semi-automated-Ansatz bezieht sich auf die Initialisierung des Trackens. Im Gegensatz zum automated tracking, in dem der Tracker mittels bereits bestehende Informationen initialisiert wird, und dem manual tracking, bei dem eine Nutzerinteraktion in jedem Bild nötig ist, wird im semi-automated-Tracking nur eine Nutzereingabe für die erste Objektdefinition benötigt.

Für eine erfolgreiche Abreitsweise jedes Trackers, egal welchen Ansatz im zugrunde liegt, sind zwei Anforderungen essentiell: Genauigkeit und Robustheit \cite{key-8}. Beide sind jedoch, bezogen auf die Möglichkeiten zur Optimierung, gegensätzlich. Die Genauigkeit wird vor allem dadurch erreicht wird, dass der Suchradius des Objekts im Folgebild möglichst gering ist, um eine Fehlerfunktion zu minimieren. Zur Optimierung der Robustheit und damit die Reaktion auf eventuelle Änderungen der Lichtverhältnisse oder Bewegungen des Objekts, sollte gerade ein großer Radius zur Suche genutzt werden, damit das Objekt beispielsweise durch eine schnelle Bewegung nicht aus dem Suchfeld verschwindet. Damit muss ein Mittelweg gefunden werden, da eine gleichzeitige Optimierung beider Anforderungen nicht möglich ist.

Typischerweise ist das Finden und Verfolgen von Objekten mit einer nicht-statitischen Kamera in der ``realen'' Welt eines der herausfordernsten Problemstellungen der Computer Vision, da die Umwelt einen großen Einfluss auf die Qualität der visuellen Daten hat. Lichtverhältnisse variieren, eine Fülle anderer Objekte hat eventuell große Ähnlichkeit zum gesuchten Objekt haben, komplexe und nichtstatische Strukturen wie Bäume erschweren das Separieren. Zusätzlich ist definiert, dass das Objekt, in diesem Fall der Kopter, keinerlei Markierungen in Form von farbigen Markern oder speziellen, eindeutigen Mustern erhält.
\end{singlespace}
\end{abstract}